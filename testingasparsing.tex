%
% The first command in your LaTeX source must be the \documentclass command.
\documentclass[sigplan,review]{acmart}

%
% \BibTeX command to typeset BibTeX logo in the docs

% Rights management information. 
% This information is sent to you when you complete the rights form.
% These commands have SAMPLE values in them; it is your responsibility as an author to replace
% the commands and values with those provided to you when you complete the rights form.
%
% These commands are for a PROCEEDINGS abstract or paper.
\copyrightyear{2019}
\acmYear{2019}
\setcopyright{acmlicensed}
\acmConference[Woodstock '18]{Woodstock '18: ACM Symposium on Neural Gaze Detection}{June 03--05, 2018}{Woodstock, NY}
\acmBooktitle{Woodstock '18: ACM Symposium on Neural Gaze Detection, June 03--05, 2018, Woodstock, NY}
\acmPrice{15.00}
\acmDOI{10.1145/1122445.1122456}
\acmISBN{978-1-4503-9999-9/18/06}

%
% These commands are for a JOURNAL article.
%\setcopyright{acmcopyright}
%\acmJournal{TOG}
%\acmYear{2018}\acmVolume{37}\acmNumber{4}\acmArticle{111}\acmMonth{8}
%\acmDOI{10.1145/1122445.1122456}

%
% Submission ID. 
% Use this when submitting an article to a sponsored event. You'll receive a unique submission ID from the organizers
% of the event, and this ID should be used as the parameter to this command.
%\acmSubmissionID{123-A56-BU3}

%
% The majority of ACM publications use numbered citations and references. If you are preparing content for an event
% sponsored by ACM SIGGRAPH, you must use the "author year" style of citations and references. Uncommenting
% the next command will enable that style.
%\citestyle{acmauthoryear}

%
% end of the preamble, start of the body of the document source.
\begin{document}


%
% The "title" command has an optional parameter, allowing the author to define a "short title" to be used in page headers.

\title{How to Assault Your Code with Mad Libs\\for Fun and Profit}

%
% The "author" command and its associated commands are used to define the authors and their affiliations.
% Of note is the shared affiliation of the first two authors, and the "authornote" and "authornotemark" commands
% used to denote shared contribution to the research.
\author{Alex Groce}
\email{agroce@gmail.com}
\affiliation{%
  \institution{School of Informatics, Computing, and Cyber Systems\\Northern Arizona University}
  \streetaddress{1295 S Knoles Dr.}
  \city{Flagstaff}
  \state{Arizona}
  \postcode{86011}
}

\author{David R. MacIver}
\email{david@drmaciver.com}
\affiliation{%
  \institution{Imperial College London}
  \city{London}
  \country{United Kingdom}}

\author{Peter Goodman}
\email{peter@trailofbits.com}
\author{Gustavo Greico}
\email{gustavo.greico@trailofbits.com}
\affiliation{%
  \institution{Trail of Bits}
  \city{New York}
  \state{New York}
}

 

%
% By default, the full list of authors will be used in the page headers. Often, this list is too long, and will overlap
% other information printed in the page headers. This command allows the author to define a more concise list
% of authors' names for this purpose.
%\renewcommand{\shortauthors}{Groce, et al.}

%
% The abstract is a short summary of the work to be presented in the article.
\begin{abstract}
We need an abstract.
\end{abstract}

%
% The code below is generated by the tool at http://dl.acm.org/ccs.cfm.
% Please copy and paste the code instead of the example below.
%
\begin{CCSXML}
<ccs2012>
<concept_id>10011007.10011074.10011099.10011102.10011103</concept_id>
<concept_desc>Software and its engineering~Software testing and debugging</concept_desc>
<concept_significance>500</concept_significance>
</concept>
</ccs2012>
\end{CCSXML}

\ccsdesc[500]{Software and its engineering~Software testing and debugging}
%
% Keywords. The author(s) should pick words that accurately describe the work being
% presented. Separate the keywords with commas.
\keywords{automated test generation, parsing, parameterized unit testing}

%
% A "teaser" image appears between the author and affiliation information and the body 
% of the document, and typically spans the page. 


%
% This command processes the author and affiliation and title information and builds
% the first part of the formatted document.
\maketitle

\section{Introduction: What is a Mad Lib?}

Software testing research has many interesting and powerful techniques for finding bugs in software.
Most of these have two things in common:
The first is that they are centered around attempting to generate \emph{test cases} which some \emph{test oracle} determines triggers a bug in some \emph{system under test} (SUT).
The second is that almost nobody uses them in practice.

As researchers in this field, it would be natural for us to object to the implied criticism there,
and to some extent we do.
The goal of a researcher is not to build tools, it is to do good research,
and the software testing research community has certainly achieved that.
If the tools are not being built, this is arguably a failure of industry and not of the research community.

This objection is reasonable, defensible, and problematic.
Although it may be unreasonable to expect researchers to build tools,
the lack of uptake from industry implies certain ``missing research questions''.
It's perhaps too much to hope for research papers to include ``\textbf{RQ1}: Why is nobody using our stuff?'',
but at the minimum we would like to see more techniques from software testing research making it into practice,
and suggest that the fact that they are not may be because we've missed out on some bridging research that would make that possible.

One area where software testing search has enjoyed success is ``property-based testing'',
a term describing a loose family of libraries designed to make it easy for practitioners to write generative tests for themselves,
originally popularised by QuickCheck, a Haskell and later Erlang library for software testing.
QuickCheck itself is quite popular,
and many libraries inspired by it exist in other languages.

The key to QuickCheck's success is mostly ease of use,
and key to that ease is that practioners can readily generate test cases which are suitable for just about any SUT or test oracle.
In particular, it is easy to \emph{compose} test-case generators,
so users can easily build test-case generators suitable for their needs.

At the heart of QuickCheck's ease of composition is the fact that it uses \emph{random test-case generation}.
Random test-case generation is one of the few methods that is widely used in practice, because it is robust and easy to implement---all
you need to know to write a tolerably effective random test-case generator is how to construct a test case of the right format.
Anyone testing their software with handwritten tests knows how to construct the relevant test cases,
so it is a relatively small step from hand writing test cases to generating them.

Set aside this is the problem that, while random testing is surprisingly effective,
it falls significantly short of the state of the art,
and the random test-case generators used in property-based testing are in turn not especially sophisticated and fall well short of the state of the art of random testing.

Unfortunately, most more advanced systems have historically lacked the pleasing compositional properties of random testing,
because they require more sophisticated operations on or understanding of the generated test cases.
We believe this is one of the key obstacles to broader adoption.

If true, this leaves us in the unfortunate situation that the systems that are easy to adopt are not especially effective at finding bugs,
and the systems that are effective at finding bugs are not ones that are easy to adopt.

In this paper we discuss one possible way to get some of the best of both worlds,
using the idea that we can view random test-case generation as a form of parsing,
and that we can manipulate the parsed input format in more interesting ways than just generating it uniformly at random.
This retains all the compositional properties of random test-case generation,
but by giving us a concrete representation we can work with opens up the possibility of bringing more advanced techniques to bear on the problem.

In this paper we will 1) Explain the underlying basis for this approach 2) Describe how it has been used in several practical tools and 3) Suggest future research directions that it might be profitable to pursue.

\section{The Simplest Random Tester}

\section{Everything is Parsing}

\subsection{A Test Generator as an Interpreter}

\section{Tools that Work This Way}

One such tool is Hypothesis \cite{hypothesis}.  Another is DeepState \cite{DeepState}.

\section{Practical Implications:  Implementing Ranges in DeepState}



% The next two lines define the bibliography style to be used, and the bibliography file.
\bibliographystyle{ACM-Reference-Format}
\bibliography{bibliography}

% 

\end{document}
